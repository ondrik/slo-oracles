\documentclass{beamer}

\mode<presentation>
{
    %\usetheme{AnnArbor}
    %\usetheme{Antibes}
    %\usetheme{Bergen}
    %\usetheme{Berkeley}
    %\usetheme{Berlin}
    %\usetheme{Boadilla}
    \usetheme{CambridgeUS}
    %\usetheme{Copenhagen}
    %\usetheme{Darmstadt}
    %\usetheme{Dresden}
    %\usetheme{Frankfurt}
    %\usetheme{Goettingen}
    %\usetheme{Hannover}
    %\usetheme{Ilmenau}
    %\usetheme{JuanLesPins}
    %\usetheme{Luebeck}
    %\usetheme{Madrid}
    %\usetheme{Malmoe}
    %\usetheme{Marburg}
    %\usetheme{Montpellier}
    %\usetheme{PaloAlto}
    %\usetheme{Rochester}
    %\usetheme{Singapore}            % maybe
    %\usetheme{Szeged}
    %\usetheme{Warsaw}
    %\setbeamercovered{transparent}
    %\usecolortheme{seahorse}
    \usecolortheme{dolphin}
    %\usecolortheme{dove}
}

\setbeamertemplate{itemize item}[square]
\setbeamertemplate{itemize subitem}[circle]
\setbeamertemplate{itemize subsubitem}[triangle]
\setbeamertemplate{enumerate item}[square]
\setbeamertemplate{section in toc}[square]
\setbeamertemplate{navigation symbols}{}

\usepackage{ucs}
\usepackage[utf8x]{inputenc}
\usepackage{palatino}
\usepackage{color}
\usepackage{graphicx}
\usepackage{alltt}
\usepackage{wasysym}

\newcommand{\X}{\mathcal{X}}
\newcommand{\Pcl}{\mathsf{\mathbf{P}}}
\newcommand{\NPcl}{\mathsf{\mathbf{NP}}}
\newcommand{\REcl}{\mathsf{\mathbf{RE}}}
\newcommand{\coREcl}{\mathsf{\mathbf{coRE}}}
\newcommand{\Rcl}{\mathsf{\mathbf{R}}}

\newcommand{\qquest}{q_{\,?}}
\newcommand{\qyes}{q_{\,\mathsf{YES}}}
\newcommand{\qno}{q_{\,\mathsf{NO}}}

\newcommand{\HP}{\mathit{HP}}
\newcommand{\coHP}{\mathit{coHP}}

\newcommand{\st}{\,|\,}

\newcommand{\hlbl}[1]{\textcolor{blue}{#1}}


\title{Oracle Machines}

\author{Complexity Theory}

\date{}

\begin{document}
  
%*******************************************************************************
\begin{frame}[plain]
  \titlepage
\end{frame}
%*******************************************************************************

%===============================================================================
\section{Oracle Machines}\label{}
%===============================================================================

%*******************************************************************************
\begin{frame}
  \frametitle{\insertsection}
  \begin{itemize}
    \item  Computation in an \hlbl{alternative universe}.
    \item  An algorithm can once in a while ask an \hlbl{oracle} a yes/no
      question and immediately get the answer.
    \item  Relations between complexity classes may be different in these
      universes,
      \begin{itemize}
        \item  later, we will show that there are oracles for which both
          \begin{equation*}
            \Pcl = \NPcl \hspace{1cm} \mbox{and} \hspace{1cm} \Pcl \neq \NPcl~.
          \end{equation*}
      \end{itemize}
  \end{itemize}
\end{frame}
%*******************************************************************************

%*******************************************************************************
\begin{frame}
  \frametitle{\insertsection}
  \begin{definition}[Turing machine with oracle]
    Given an arbitrary language $\X \subseteq \Sigma^{*}$, a \hlbl{Turing
    machine with oracle $M^{\X}$} (or an \hlbl{Oracle machine}) is a multi-tape
    deterministic Turing machine with a special \hlbl{query tape} and three
    special states: $\qquest$, the \hlbl{query state}, and $\qyes, \qno$, the
    \hlbl{answer states}.
    The \hlbl{computation} of $M^{\X}$ proceeds like in an ordinary TM,
    except from transitions from $\qquest$.
    Let $w \in \Sigma^{*}$ be the content of the query tape, then
    $M^{\X}$ moves from $\qquest$ to $\qyes$ iff $w \in \X$ and to $\qno$
    otherwise.
  \end{definition}

  \vspace{3mm}
  Similarly for \hlbl{nondeterministic Turing machines with oracles}.


\end{frame}
%*******************************************************************************

%*******************************************************************************
\begin{frame}
  \frametitle{\insertsection}
  Time complexity for oracle machines is defined in the same way as time
  complexity for Turing machines,
  \begin{itemize}
    \item  each \hlbl{query step} counts as one ordinary step,
    \item  this can make oracle machines \hlbl{very} powerful.
  \end{itemize}

  \vspace{3mm}
  Oracle machines have many uses in the theory of \hlbl{computational
  complexity} and \hlbl{decidability}.
\end{frame}
%*******************************************************************************

%+++++++++++++++++++++++++++++++++++++++++++++++++++++++++++++++++++++++++++++++
\subsection{Oracle Machines and Decidability}\label{}
%+++++++++++++++++++++++++++++++++++++++++++++++++++++++++++++++++++++++++++++++

%*******************************************************************************
\begin{frame}
  \frametitle{\insertsubsection}
  Consider the following two familiar languages:
  \begin{align*}
      \HP_1 &= \{<\!M;x\!> \st \mbox{TM $M$ halts on $x$}\} \\
    \coHP_1 &= \{<\!M;x\!> \st \mbox{TM $M$ does not halt on $x$}\}
  \end{align*}
  \vspace{3mm}
  Note that:
  \begin{itemize}
    \item  $\HP_1 \in \REcl \setminus \Rcl$ and $\coHP_1 \in \coREcl \setminus
      \Rcl$,
    \item  further, $\HP_1$ is \hlbl{complete} for $\REcl$ and $\coHP_1$ is
      \hlbl{complete} for $\coREcl$.
  \end{itemize}
  \vspace{5mm}
  \pause
  Now, consider the following Oracle machine: $M^{\HP_1}$,
  \begin{itemize}
    \item  obviously, $M^{\HP_1}$ is able to \hlbl{decide} both $\HP_1$ and
      $\coHP_1$.
  \end{itemize}
\end{frame}
%*******************************************************************************

%*******************************************************************************
\begin{frame}
  \frametitle{\insertsubsection}
  However, consider the following two languages:
  \begin{equation*}
  \begin{array}{rll}
      \HP_2 &= \{<\!M^{\HP_1};x\!>   & \st \mbox{OM\, $M^{\HP_1}$ halts on $x$}\} \\
    \coHP_2 &= \{<\!M^{\coHP_1};x\!> & \st \mbox{OM\, $M^{\coHP_1}$ does not halt on $x$}\}
  \end{array}
  \end{equation*}
  \pause
  \begin{itemize}
    \item  It is easy to see that $M^{\HP_1}$ is able to \hlbl{accept} $\HP_2$,
    \item  nonetheless, $M^{\HP_1}$ \alert{can neither}
      \begin{equation*}
        \mbox{\hlbl{decide}}~\HP_2\hspace{1cm}\mbox{\alert{nor}}\hspace{1cm}
        \mbox{\hlbl{accept}}~\coHP_2,
      \end{equation*}
    \item  this can be also shown using \hlbl{diagonalisation}.
  \end{itemize}
  \pause
  \vspace{2mm}
  Further, one can create $M^{\HP_2}$,
  \begin{itemize}
    \item  for which there exist $\HP_3$ and $\coHP_3$,
    \item  and so on \dots
  \end{itemize}
\end{frame}
%*******************************************************************************

%*******************************************************************************
\begin{frame}
  \frametitle{\insertsubsection}
  \begin{definition}[Recursive relation]
    Relation $R \subseteq \left(\Sigma^{*}\right)^k$ is \hlbl{recursive} if the
    language $L_R$ is recursive,
    \vspace{-2mm}
    \begin{equation*}
    L_R = \{x_1; \dots ; x_k \st (x_1, \dots, x_k) \in R\}~.
    \end{equation*}
  \end{definition}

  \pause
  \begin{definition}[Arithmetical hierarchy]
    For $k \geq 0$, let \hlbl{$\Sigma_k$} be the class of all languages $L$ for
    which there is a recursive $(k+1)$-ary relation $R$ such that
    \vspace{-2mm}
    \begin{equation*}
      L = \{x \st \exists x_1 \forall x_2 \dots Q_k x_k\,.\,(x_1, \dots, x_k, x)
      \in R\}~.
    \vspace{-2mm}
    \end{equation*}
%    Further, let \hlbl{$\Pi_k$} $= \mathsf{\mathbf{co}}\Sigma_k$, or, equivalently,
    Let \hlbl{$\Pi_k$} be the class of languages $L$ for which there is $R$ s.t.
    \vspace{-2mm}
    \begin{equation*}
      L = \{x \st \forall x_1 \exists x_2 \dots Q_k x_k\,.\,(x_1, \dots, x_k, x)
      \in R\}~.
    \vspace{-2mm}
    \end{equation*}
    The family $\left\{\Sigma_k\st k\geq 0\right\}$ is called the
    \hlbl{arithmetical hierarchy}.
  \end{definition}
  \pause
  Note that for $k \geq 0$, it holds that $\Pi_k = \mathsf{\mathbf{co}}\Sigma_k$.
\end{frame}
%*******************************************************************************

%*******************************************************************************
\begin{frame}
  \frametitle{\insertsubsection}
  Interesting properties:
  \begin{itemize}
    \item  $\Sigma_0 = \Pi_0 = \Rcl$,
    \item  $\Sigma_1 = \REcl$, and $\Pi_1 = \coREcl$,
    \item  $\Sigma_{k+1} \supset \Sigma_k, \Pi_k$, and $\Pi_{k+1} \supset
      \Sigma_k, \Pi_k$.
  \end{itemize}
  \vspace{3mm}
  Examples of problems and their membership:
  \begin{itemize}
    \item  $\HP_1 \in \Sigma_1$, and $\coHP_1 \in \Pi_1$,
    \item  $\HP_2 \in \Sigma_2$, and $\coHP_2 \in \Pi_2 \setminus \Pi_1$,
    \item  \dots
    \item  $\left\{<\!M\!>\st L(M) = \Sigma^{*}\right\} \in \Pi_2 \setminus
      \Pi_1$,
    \item  $\left\{<\!M\!>\st L(M)~\mbox{is infinite}\right\} \in ???$.
  \end{itemize}
  \vspace{3mm}
  AH enables finer classification of undecidable problems.
\end{frame}
%*******************************************************************************

%*******************************************************************************
\begin{frame}
  \frametitle{\insertsubsection}
  finer classification of undecidable problems
\end{frame}
%*******************************************************************************

%*******************************************************************************
\begin{frame}
  \frametitle{Oracle Machines and Function Problems}
\end{frame}
%*******************************************************************************

\end{document}
